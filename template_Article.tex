\documentclass[titlepage]{article}
\usepackage{flexisym}

%opening
\title{Computability and Computational Complexity Exercises Explained}
\author{Michael Dallago}
\date{October-December 2016}

\newcounter{exercise}
\newcounter{solution}

\newcommand\Exercise{%
	\stepcounter{exercise}%
	\textbf{Exercise \theexercise.}~%
	\setcounter{solution}{0}%
}

\newcommand\TheSolution{%
	\textbf{Solution:}\\%
}

\newcommand\ASolution{%
	\stepcounter{solution}%
	\textbf{Solution \thesolution:}\\%
}
\parindent 0in
\parskip 1em
\begin{document}

\maketitle

\section{Undecidability}
\Exercise  Give informal reductions from the “hello-world” problem to the following:
\begin{enumerate}
	\item Given a program and an input, does the program stop?
	\item Given a program and an input, does the program ever produce an
	output?
	\item Given two programs and an input, do the two programs produce the
	same output?
\end{enumerate}

\TheSolution \begin{enumerate}
	\item In order to understand whether the program \textit{P} stops on input \textit{I}, modify \textit{P} constructing a new program $P^\prime$ so that
	\begin{itemize}
		\item When \textit{P} would halt, $P^\prime$ will output \texttt{hello, world}
		\item When \textit{P} would output \texttt{hello, world}, $P^\prime$ halt
	\end{itemize}
	In this way, if the original program $P$ printed \texttt{hello, world}, the new $P^\prime$ will halt, whereas if the original program stopped, the new one will print \texttt{hello, world}.
	
	\item This is simply accomplished by replacing any output statement of $P$ by one that outputs \texttt{hello, world}. 
	
	\item If we consider $ P $ as a general program and $ P^\prime $ as the standard \texttt{hello, world} one, then checking whether the two programs produce the same output is trivial, since you just need to replace any output of $ P $ by one that outputs \texttt{hello, world}.
\end{enumerate}

\clearpage

\section{Turing Machines}

\setcounter{exercise}{0}
\Exercise Show the Instantaneous Descriptions (IDs) of the Turing Machine for $ \{0^{n}1^{n}\} $ for the following input tapes:
\begin{enumerate}
	\item 00
	\item 000111
	\item 00111
\end{enumerate}

\TheSolution


\end{document}
